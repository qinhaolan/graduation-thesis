\chapter{总结与展望}
\par 
本文以模糊理论和熵理论为基础,研究了模糊熵在最大熵准则的约束下,在聚类分析中的应用,并建立了模糊最大熵模型.

在前面,我们首先介绍了模糊集和最大熵原理.独立的系统演化是一个熵增的过程,在没有外力的作用下,熵是一直增加的,即符合我们的最大熵原理.
而自然界也是一个巨大的系统,时时刻刻产生信息,有精确的,但许多都是模糊的.
进一步,为了在已有的知识下,使我们的结果更加准确,即符合自然分布,我们趋向于使得信息的熵最大,于是我们在FCM的基础上融入了最大熵模型,形成了模糊最大熵模型.
最后我们将模糊最大熵模型应用于iris数据集和seeds小麦种子的分类,取得了相对于Kmeans聚类和FCM聚类更好的聚类效果.
\par
对于下一步的探讨,我觉得是在差异因子$\beta$ 的选取上,寻找更好的算法选取差异因子$\beta$ ,而不是简简单单使用数据集的平均值代替,可能会有更好的聚类效果.
另一个思路是选取不同的模糊熵计算公式,对不同的模糊熵最大化聚类效果进行比较.
\par
现如今,适逢新一代人工智能的浪潮,各种智能算法、机器学习研究论文层出不穷,而模糊聚类在图像分割、目标识别、故障诊断等方面也有广泛的应用,
相信在不久的未来,关于模糊最大熵模型的应用与机器学习的结合会越来越多,推动模糊分类算法变得越来越好.