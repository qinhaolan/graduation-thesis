\begin{cnabstract}
本文先介绍了模糊数学中常用的模糊C均值(Fuzzy C-means.FCM)算法.
在模糊C均值算法中,通过与模糊数学的融合,给出了相比于K-means硬聚类更灵活的聚类结果.
在FCM算法的基础上,我们引入最大熵准则构建模糊最大熵模型,在UCI的iris数据集和seeds数据集上应用了我们的模型,并比较了它们的分类准确率.
对于非线性数据,我们的模型相对于传统的Kmeans聚类和FCM聚类有更好的表现.
  \cnkeywords{模糊集,熵,最大模糊熵,模糊聚类}
\end{cnabstract}
\renewcommand{\abstractname}{Abstract}
\begin{abstract}
  This thesis first introduces the fuzzy C-means (FCM) algorithm which is commonly used in fuzzy mathematics.
In the fuzzy C-means algorithm, by integrating with fuzzy mathematics, it gives more flexible clustering results compared with K-means hard clustering.
Based on the FCM algorithm, we introduced the maximum entropy criterion to construct the fuzzy maximum entropy model, applied our fuzzy maximum entropy model on the iris dataset and seeds dataset of UCI, and compared their classification accuracies.
For nonlinear data, our fuzzy maximum entropy model has better performance compared with traditional Kmeans clustering and FCM clustering.
  \keywords{fuzzy sets;entropy;maximum fuzzy entropy;fuzzy clustering}
\end{abstract} 