\begin{cnabstract}
本文先介绍了模糊数学中常用的模糊C均值(Fuzzy C-means.FCM)算法,在模糊C均值算法中,通过与模糊数学的融合,给出了相比于K-means硬聚类更灵活的聚类结果.
在FCM算法的基础上,我们介绍了一种基于最大熵准则的模型:模糊最大熵模型.对于非线性数据,我们的模糊最大熵模型相对于传统的Kmeans聚类和FCM聚类有更好的表现.
最后我们在UCI的iris数据集h和seeds数据集上应用了我们的最大模糊熵模型,并比较了它们的分类准确率.
  \cnkeywords{模糊集,熵,最大模糊熵,模糊聚类}
\end{cnabstract}
\renewcommand{\abstractname}{Abstract}
\begin{abstract}
  In this paper, Fuzzy C-Means.FCM algorithm, which is commonly used in Fuzzy mathematics, is introduced firstly.
   In the Fuzzy C-Means algorithm, the clustering result is more flexible than K-Means hard clustering by the fusion with Fuzzy mathematics.
   On the basis of FCM algorithm, we introduce a model based on maximum entropy criterion: fuzzy maximum entropy model.
   For nonlinear data, our fuzzy maximum entropy model has a better performance than traditional Kmeans clustering and FCM clustering.
   Finally, we applied our maximum fuzzy entropy model to the IRIS dataset H and SEEDS dataset of UCI, and compared their classification accuracy.
  \keywords{fuzzy sets;entropy;maximum fuzzy entropy;fuzzy clustering}
\end{abstract} 