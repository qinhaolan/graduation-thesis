\chapter[模糊最大熵准则的模糊聚类模型]{模糊最大熵准则的模糊聚类}

\section{最大熵原理}
熵原本是物理学中的概念,是由热力学第二定律引出的一个物质系统的状态参量,是反映系统的混乱程度,度量信息有效性的一个重要工具.
自从熵的概念被提出来之后,很快引起了其他领域研究者的注意,熵理论得以迅速传播,渗透进各个领域.
\subsection{信息熵}
1948年,香农\cite{1948A}提出了信息熵的概念,解决了信息的量化度量问题.\\
(1)离散模糊变量的信息熵\\
设某个离散型的随机变量 $X, \quad X$ 的分布率是 $\left\{p_{i}\right\}$, 且 $p_{i}=P\left\{X=x_{i}\right\}, 0 \leq p_{i} \leq 1$,
$\sum_{i=1}^n \mathrm{p}_i=1, \quad(i=1,2 ,\dots, n)$ 则用信息熵来度量事件 $X$ 的确定程度为:
\begin{equation}
    H(X) =-\sum_{i=1}^{n} p_{i} \operatorname{\ln p}_{i}
\end{equation}
(2)连续连续模糊变量的信息熵\\
假设某个连续型随机变量 $X$ ,其概率密度函数是 $p(x)$, 则该连续随机变量 $X$ 的信
息熵为:
\begin{equation}
    H(X)=-\int p(x) \ln p(x) d x
\end{equation}

\subsection{最大熵}
1957年,E.T.Jaynes提出了最大熵原理,通过把随机变量与信息熵联系起来,然后最大化信息熵.
最大熵原理并不是某一固定的数学公式,而是一种选择随机变量的准则.
最大熵的主要思想是,在我们只掌握未知变量的部分知识的时候,未知变量的分布可能有很多种,此时我们应该选符合这些知识的情况下,使得信息熵取得最大值的概率分布.

\section{模糊熵}
在自然界以及我们的日常生活中,经常存在着带着模糊性质的不确定现象,我们把它们定义成模糊变量.
自从模糊理论提出以后,很多学者就在如何度量模糊变量的模糊程度这一方面进行了许多研究,
比如Li, Pingke 和Liu, Baoding\cite{li2008entropy}、Aldo de Luca 和 Settimo Termini\cite{RN3}等.
\begin{definition}[模糊熵]
    对于离散变量的模糊
    \begin{equation}
        H(\tilde{A})=-\sum_{i=1}^{n}\tilde{A}(u_i) \ln \tilde{A}(u_i)
    \end{equation}

    对于连续的模糊变量
    \begin{equation}
        H(\tilde{A})=-\int_{-\infty}^{\infty}\tilde{A}(u) \ln \tilde{A}(u) d u
    \end{equation}

\end{definition}

\newpage
\section{模糊最大熵模型}
定义了模糊熵之后,我们就可以将最大熵原理推广到模糊熵的情形.
设$a_{i j}$为第$j$个元素对于第$i$类的隶属度,则我们模糊最大熵模型(Fuzzy Maximum Entropy.FME)的目标函数可以表示成
\begin{equation}
    \max \biggl\{-\sum_{j=1}^{n} \sum_{i=1}^{c} a_{i j}\ln a_{i j}\biggr\}
    \label{MEC}
\end{equation}
我们首先得定义我们的损失函数,在这里,我们选取
\begin{equation*}
    \mathrm{L}=\sum_{i=1}^{c} \sum_{j=1}^{n}a_{i j} d_{i j}^2
\end{equation*}

作为我们的损失函数.
于是我们可以将问题归结为最优化问题,用拉格朗日乘子法:
\begin{equation}
    L(A,\beta ,\lambda)=\sum_{i=1}^{c} \sum_{j=1}^{n} a_{i j} d_{i j}^2+\beta \sum_{i=1}^{c} \sum_{i=1}^{n} a_{i j} \ln a_{i j}+\sum_{j=1}^{n} \lambda_{j}\left(\sum_{i=1}^{c} a_{i j}-1\right)
    \label{MLagrange}
\end{equation}

其中$\beta$是差异因子,由数据集的分布情况来决定,$\lambda_j$是$\sum\limits_{i=1}^{c} a_{i j}$的拉格朗日乘子.
\begin{equation}
    \frac{\partial L(A,\beta ,\lambda) }{\partial a_{i j}} =d_{i j}^2+\beta(\ln a_{i j}+1)+\lambda_j=0
\end{equation}
\begin{equation}
    \frac{\partial L(A,\beta ,\lambda)}{\partial \lambda_j}=\sum_{i=1}^{c} a_{i j}-1=0
\end{equation}
最后解得
\begin{equation}
    a_{i j}=\frac{ \exp(\frac{-d_{i j}^2}{\beta})}{\sum\limits_{i=1}^c\exp(\frac{-d_{i j}^2}{\beta})}
    \label{Maij}
\end{equation}
\begin{equation}
    v_i =\frac{\sum\limits_{j=1}^n a_{i j} x_j}{\sum\limits_{j=1}^n a_{i j}}
    \label{Mvij}
\end{equation}
