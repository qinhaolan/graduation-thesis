\chapter[模糊最大熵准则的聚类模型]{模糊最大熵准则的聚类模型}

\section{模糊最大熵模型}
前面我们介绍了模糊熵和最大熵原理,还介绍了模糊聚类的一种常用方法:模糊C-均值聚类.
现在,我们就可以将最大熵原理推广到模糊熵的情形,在模糊C-均值聚类中加入模糊熵进行最大熵约束.
\par
设$a_{i j}$为第$j$个元素对于第$i$类的隶属度,则我们模糊最大熵模型(Fuzzy Maximum Entropy.FME)的目标函数可以表示成
\begin{equation}
    \max \biggl\{-\sum_{j=1}^{n} \sum_{i=1}^{c} a_{i j}\ln a_{i j}\biggr\}
    \label{MEC}
\end{equation}
我们首先得定义我们的损失函数,在这里,我们选取
\begin{equation*}
    \mathrm{L}=\sum_{i=1}^{c} \sum_{j=1}^{n}a_{i j} d_{i j}^2
\end{equation*}

作为我们的损失函数.
于是我们可以将问题归结为最优化问题,用拉格朗日乘子法:
\begin{equation}
    L(A,\beta ,\lambda)=\sum_{i=1}^{c} \sum_{j=1}^{n} a_{i j} d_{i j}^2+\beta \sum_{i=1}^{c} \sum_{i=1}^{n} a_{i j} \ln a_{i j}+\sum_{j=1}^{n} \lambda_{j}\left(\sum_{i=1}^{c} a_{i j}-1\right)
    \label{MLagrange}
\end{equation}

其中$\beta$是差异因子,由数据集的分布情况来决定,$\lambda_j$是$\sum\limits_{i=1}^{c} a_{i j}$的拉格朗日乘子.
\begin{equation}
    \frac{\partial L(A,\beta ,\lambda) }{\partial a_{i j}} =d_{i j}^2+\beta(\ln a_{i j}+1)+\lambda_j=0
\end{equation}
\begin{equation}
    \frac{\partial L(A,\beta ,\lambda)}{\partial \lambda_j}=\sum_{i=1}^{c} a_{i j}-1=0
\end{equation}
最后解得
\begin{equation}
    a_{i j}=\frac{ \exp(\frac{-d_{i j}^2}{\beta})}{\sum\limits_{i=1}^c\exp(\frac{-d_{i j}^2}{\beta})}
    \label{Maij}
\end{equation}
\begin{equation}
    v_i =\frac{\sum\limits_{j=1}^n a_{i j} x_j}{\sum\limits_{j=1}^n a_{i j}}
    \label{Mvij}
\end{equation}

\section{模糊最大熵模型的求解算法}
在描述我们的算法之前,我们首先来定义一些模型的输入输出:
\[
    \text{n个样本的m维数据集}U:U=\{u_1,u_2,\dots ,u_n\} ,u_i=\{x_1,x_2,\dots,x_m\} \forall i \\
\]
\[
    \text{聚类数}c:2\leqslant c \leqslant n
\]
\[
    \text{隶属度矩阵}A:n\times m\text{的矩阵}
\]
\[
    \text{聚类中心}V:c\times m\text{的矩阵}
\]
对于参数$\beta$,我们可以用数据集的平均值计算:
\begin{equation}
    \beta=\frac{1}{n}\sum_{i=1}^{n} \frac{1}{m}\sum_{j=1}^{m} x_{i j}
    \label{beta}
\end{equation}
模糊最大熵模型的聚类算法步骤如下:\\
输入:原始数据$U$和聚类数$c$\\
输出:隶属度矩阵$A$和聚类中心$V$
\begin{itemize}
    \item[\bf{1)}]首先随机初始化隶属度矩阵并归一化;
    \item[\bf{2)}]用式\ref{beta}计算参数$\beta$;
    \item[\bf{3)}]根据式\ref{Mvij}计算聚类中心;
    \item[\bf{4)}]计算每一个元素到聚类中心的距离;
    \item[\bf{5)}]根据式\ref{Maij}更新隶属度矩阵;
    \item[\bf{6)}]如果到达指定精度或迭代次数,结束计算过程,否则重复$3\sim 5$步;
    \item[\bf{7)}]输出隶属度矩阵$A$和聚类中心$V$.
\end{itemize}
\par 
此算法类似于模糊ISODATA算法,都是在循环迭代中,通过隶属度公式和聚类中心公式断更新当前值,最后达到最优解.