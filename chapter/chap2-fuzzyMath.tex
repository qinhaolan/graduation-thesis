%TEX root=../main.tex
\chapter[模糊数学理论]{模糊数学理论}
\section{模糊集的概念}
\par
在经典集合理论里面,一个集合就是某一个概念的内涵。
对于论域上的一个对象,它要么属于这个集合,要么不属于这个集合,两者只能选一个,不能两者兼之,也不能有模棱两可的情况。
而模糊数学研究的对象来说,我们不能简单地用是或否来描述一个对象是否属于一个集合,给对象引入了隶属度这一个概念,从而给经典集合论里面引入了模糊性。
概括地说,模糊数学就是将现实世界的模糊概念抽象出来作为研究对象,然后用精确的数学方法探寻其中的数量规律的一门数学分支。

\section{模糊集及其表示方法}
模糊集,又叫模糊子集
\begin{definition}[模糊集\scite{ZADEH1965fuzzy}]
    设$\mathrm{U}$为自论域,
    \[
        \mu_{\tilde{A}}: \mathrm{U} \longrightarrow[0,1]
    \]
    一个
\end{definition}