%TEX root=../main.tex
\chapter[模糊数学理论]{模糊数学理论}
\par
在经典集合理论里面,一个集合就是某一个概念的内涵。
对于论域上的一个对象,它要么属于这个集合,要么不属于这个集合,两者只能选一个,不能两者兼之,也不能有模棱两可的情况。
而对模糊数学研究的对象来说,我们不能简单地用是或否来描述一个对象是否属于一个集合。
由此,我们把集合的特征函数的取值从$\{0,1\}$这个集合扩充到$[0,1]$这个区间上的连续取值。
越靠近1,说明该对象属于集合的程度越大,反之,越靠近0就越小。
%概括地说,模糊数学就是将现实世界的模糊概念抽象出来作为研究对象,然后用精确的数学方法探寻其中的数量规律的一门数学分支。
这样我们就把经典集合扩充到带有模糊边界的模糊集了,从而我们可以用这样的集合表示模糊概念。
\section{模糊集及其表示方法}
\subsection{模糊集的定义}
\begin{definition}[模糊子集\scite{ZADEH1965fuzzy}]
    设$\mathrm{U}$为我们所研究的论域,
    \[
        \mu_{\tilde{A}}: \mathrm{U} \longrightarrow[0,1]
    \]
    称$\mu$确定了$\mathrm{U}$上的一个模糊子集,记为$\tilde{A}$。
    $\mu$称为$\tilde{A}$的隶属函数,把$\mu_{\tilde{A}}(u)(u \in \mathrm{U})$的值称为$u$对于模糊子集$\tilde{A}$的隶属度。
    $\mu_{\tilde{A}}(u)$越大,代表$u$隶属于$\tilde{A}$的程度越高。
    通常,我们也把模糊子集简称为模糊集。
\end{definition}
\subsection{模糊集的表示方法}
设有限集$\mathrm{U}=\{u_1,u_2,\dots ,u_n\}$,则有限集可以用如下几种方法表示\cite{模糊数学李安贵}。
\begin{itemize}
    \item Zadeh表示法
          \[
              \tilde{A}=\frac{\tilde{A}(u_1)}{u_1}+\frac{\tilde{A}(u_2)}{u_2}+ \dots +\frac{\tilde{A}(u_n)}{u_n}.
          \]
          虽然我们以分式和的方式表示,但是其中的$\tilde{A}(u_i)/u_i$并不表示分数,“+”也不表示和。
          $\tilde{A}(u_i)/u_i$表示的是元素$u_i$与对$\tilde{A}$的隶属度的一一对应关系;“+”表示的是$\tilde{A}$在论域$\mathrm{U}$上的整体。
    \item 序偶表示法
          \[
              \tilde{A}=\{(\tilde{A}(u_1),u_1),(\tilde{A}(u_2),u_2),\dots ,(\tilde{A}(u_n),u_n)\}.
          \]
          序偶表示法是从例举法演变而来,由元素的隶属度和对应的元素组成的有序对列出。
    \item 向量表示法
          \[
              \tilde{A}=(\tilde{A}(u_1),\tilde{A}(u_2),\dots ,\tilde{A}(u_n)).
          \]
          向量表示法是用n维数组来实现的,在论域中的元素按一定的顺序排列时,按此顺序记录元素的隶属度。
          此时也称$\tilde{A}$为模糊向量。
\end{itemize}
\section{模糊集的运算及其性质}
我们先给出模糊幂集的定义:
\begin{definition}
    论域$\mathrm{U}$上的模糊子集的全体称为模糊幂集,记为$\mathscr{F}(U)$,即
    \[
        \mathscr{F}(U)=\{\tilde{A} \mid \tilde{A}(u):\mathrm{U} \to [0,1]\}
    \]
\end{definition}
模糊集的包含与相等:
\begin{definition}
    设$\tilde{A}, \tilde{B} \in \mathscr{F}(U)$,如果对$\forall u \in \mathrm{U}$
    都成立$\tilde{B}(u)\geqslant \tilde{A}(u)$,则称$\tilde{B}$包含$\tilde{A}(u)$,记作$\tilde{B}(u )\supseteq \tilde{A}(u)$。
\end{definition}
\begin{definition}
    设$\tilde{A}, \tilde{B} \in \mathscr{F}(U)$,如果对$\forall u \in \mathrm{U}$
    都成立$\tilde{B}(u) = \tilde{A}(u)$,则称$\tilde{B}$等于$\tilde{A}(u)$,记作$\tilde{B}(u)= \tilde{A}(u)$。
\end{definition}
我们规定$a\vee b=MAX(a,b),a\wedge b=MIN(a,b)$,所以我们可以这样描述模糊集的并、交、余:

\begin{definition}
    如果对于任意一个$u \in \mathrm{U}$,有$\tilde{C}(u)=\tilde{A}(u) \vee \tilde{B}$,
    则称$\tilde{C}$为$\tilde{A}$与$\tilde{B}(u)$的并,记为$\tilde{C}=\tilde{A} \cup \tilde{B}$。
    如果对于任意一个$u \in \mathrm{U}$,有$\tilde{C}(u)=\tilde{A}(u) \wedge \tilde{B}$,
    则称$\tilde{C}$为$\tilde{A}$与$\tilde{B}(u)$的交,记为$\tilde{C}=\tilde{A} \cap \tilde{B}$。\\
    它们的隶属度函数定义为:
    \[
        (\tilde{A}\cup  \tilde{B})(u) \stackrel{\text { def }}{=}\tilde{A}(u) \vee \tilde{B}(u) , \forall u \in \mathrm{U}
    \]
    \[
        (\tilde{A}\cap   \tilde{B})(u) \stackrel{\text { def }}{=}\tilde{A}(u) \wedge \tilde{B}(u) ,\forall u \in \mathrm{U}
    \]
\end{definition}
\begin{definition}
    如果对于$\forall u \in \mathrm{U}$,有$\tilde{B}(u)=1-\tilde{A}(u) $,
    则称$\tilde{B}$为$\tilde{A}$的余,记为$\tilde{B}=\tilde{A}^c$。
\end{definition}
\section{模糊集的截集}
