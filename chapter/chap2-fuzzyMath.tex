\chapter[模糊数学理论]{模糊数学理论概述}
\section{模糊数学概述}
\par
在经典集合理论里面,一个集合就是某一个概念的内涵,对于论域上的一个对象,它要么属于这个集合,要么不属于这个集合,两者只能选一个,不能两者兼之,也不能有模棱两可的情况。
而模糊数学给对象引入了隶属度这一个概念,从而给经典集合论里面引入了模糊性。
概括地说,模糊数学就是将现实世界的模糊概念抽象出来作为研究对象,然后用精确的数学方法探寻其中的数量规律的一门数学分支。
1965年,L.A.Zadeh在期刊Information and Control上发表了论文《Fuzzy Sets》\scite{ZADEH1965338},标志着模糊理论的诞生。

\section{模糊集及其表示方法}
\begin{definition}[集合]
设 $A \in \mathscr{T}(U), U$ 自论域,
\[
\begin{aligned}
\chi_{A}: U \rightarrow\{0,1\} \\
x & \rightarrow \chi_{A}(x)=\left\{\begin{array}{ll}
1, & x \in A, \\
0, & x \in A,
\end{array}\right.
\end{aligned}
\]
一个
\end{definition} 