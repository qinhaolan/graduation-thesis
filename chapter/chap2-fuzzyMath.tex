%TEX root=../main.tex
\chapter[模糊C-均值聚类]{模糊C-均值聚类}

\section{模糊集及其表示方法}
\par
在经典集合理论里面,一个集合就是某一个概念的内涵。
对于论域上的一个对象,它要么属于这个集合,要么不属于这个集合,两者只能选一个,不能两者兼之,也不能有模棱两可的情况。
而对模糊数学研究的对象来说,我们不能简单地用是或否来描述一个对象是否属于一个集合。
由此,我们把集合的特征函数的取值从$\{0,1\}$这个集合扩充到$[0,1]$这个区间上的连续取值。
越靠近1,说明该对象属于集合的程度越大,反之,越靠近0就越小。
%概括地说,模糊数学就是将现实世界的模糊概念抽象出来作为研究对象,然后用精确的数学方法探寻其中的数量规律的一门数学分支。
这样我们就把经典集合扩充到带有模糊边界的模糊集了,从而我们可以用这样的集合表示模糊概念。
\subsection{模糊集的定义}
\begin{definition}[模糊子集\scite{ZADEH1965fuzzy}]
    设$\mathrm{U}$为我们所研究的论域,
    \[
        \mu_{\tilde{A}}: \mathrm{U} \longrightarrow[0,1]
    \]
    称$\mu$确定了$\mathrm{U}$上的一个模糊子集,记为$\tilde{A}$。
    $\mu$称为$\tilde{A}$的隶属函数,把$\mu_{\tilde{A}}(u)(u \in \mathrm{U})$的值称为$u$对于模糊子集$\tilde{A}$的隶属度。
    $\mu_{\tilde{A}}(u)$越大,代表$u$隶属于$\tilde{A}$的程度越高。
    通常,我们也把模糊子集简称为模糊集。
\end{definition}
\subsection{模糊集的表示方法}
设有限集$\mathrm{U}=\{u_1,u_2,\dots ,u_n\}$,则有限集可以用如下几种方法表示\cite{模糊数学李安贵}。
\begin{itemize}
    \item Zadeh表示法
          \[
              \tilde{A}=\frac{\tilde{A}(u_1)}{u_1}+\frac{\tilde{A}(u_2)}{u_2}+ \dots +\frac{\tilde{A}(u_n)}{u_n}.
          \]
          虽然我们以分式和的方式表示,但是其中的$\tilde{A}(u_i)/u_i$并不表示分数,“+”也不表示和。
          $\tilde{A}(u_i)/u_i$表示的是元素$u_i$与对$\tilde{A}$的隶属度的一一对应关系;“+”表示的是$\tilde{A}$在论域$\mathrm{U}$上的整体。
    \item 序偶表示法
          \[
              \tilde{A}=\{(\tilde{A}(u_1),u_1),(\tilde{A}(u_2),u_2),\dots ,(\tilde{A}(u_n),u_n)\}.
          \]
          序偶表示法是从例举法演变而来,由元素的隶属度和对应的元素组成的有序对列出。
    \item 向量表示法
          \[
              \tilde{A}=(\tilde{A}(u_1),\tilde{A}(u_2),\dots ,\tilde{A}(u_n)).
          \]
          向量表示法是用n维数组来实现的,在论域中的元素按一定的顺序排列时,按此顺序记录元素的隶属度。
          此时也称$\tilde{A}$为模糊向量。
\end{itemize}
\section{模糊C-均值算法}
C-均值聚类是我们聚类经常用的方法之一,通过迭代计算使得目标函数达到局部最小值的时候,就是我们的最优分类。
在模糊C-均值聚类中,我们定义目标函数为:
\begin{equation}
    J(A, V)=\sum_{i=1}^{c} \sum_{j=1}^{n}\left(a_{i j}\right)^{r} d_{i j}^2
\end{equation}
$U=\{u_1,u_2,\dots,u_n\},u_j=(x_{j1},x_{j1},\dots,x_{jm})\in R^m$为给定的$n$个样本的$m$维数据集,
$A=(a_{i j})$是隶属度矩阵,$r$是模糊数,$d_{i k}=\| u_k-v_i\|$是第个$k$个样本到第$i$个聚类中心的距离。
\par
当$v_i$不变时问题等价于
\begin{equation}
    \min L(A, \lambda)=\sum_{i=1}^{c} \sum_{j=1}^{n}\left(a_{i j}\right)^{r}d_{i k}
\end{equation}
\begin{equation}
    s.t.  \sum_{i=i}^c a_{i k}=1.\  \forall k
\end{equation}
这是最优化问题,我们引入拉格朗日乘子$\lambda$,于是变为
\begin{equation}
    L(A, \lambda)=\sum_{i=1}^{c} \sum_{j=1}^{n}\left(a_{i j}\right)^{r}\left\|u_{j}-v_{i}\right\|^{2}-\sum_{j=1}^{n} \lambda_{j}\left(\sum_{i=1}^{c} a_{i j}-1\right)
    \label{Lagrange}
\end{equation}
对式\ref{Lagrange}求导,局部最小值时必要条件为
\begin{equation}
    \frac{\partial L(A, \lambda)}{\partial a_{i j}}=\left[r\left(a_{i j}\right)^{r-1}\left\|u_{j}-v_{i}\right\|^{2}-\lambda_{j}\right]=0
    \label{dLdAij}
\end{equation}
\begin{equation}
    \frac{\partial L(A, \lambda)}{\partial \lambda_{j}}=\sum_{i=1}^{c} a_{i j}-1=0
    \label{dLdLamda}
\end{equation}

由式\ref{dLdAij}可得:
\begin{equation}
    a_{i j}=\left( \frac{\lambda_j}{r \|u_{j}-v_{i}\|^2} \right)^{\frac{1}{r-1}}
    \label{daij}
\end{equation}

将式\ref{daij}带入式\ref{dLdLamda}解得:
\begin{equation}
    \left(\frac{\lambda_j}{r}\right)^{\frac{1}{r-1}}=\left[\sum\limits_{i=1}^c( \frac{1}{r \|u_{j}-v_{i}\|^2})^{\frac{1}{r-1}} \right]^{-1}
    \label{lambda}
\end{equation}
\newpage
最后将式\ref{lambda}代回式\ref{daij}得到隶属度的更新公式为:
\begin{equation}
    a_{i j}=\left[\sum\limits_{j=1}^c\left( \frac{\|u_{j}-v_{i}\|}{\|u_{j}-v_{j}\|} \right)^{\frac{2}{r-1}}\right]^{-1}
    \quad 1 \leqslant i \leqslant c,\quad 1 \leqslant j \leqslant n
    \label{aij}
\end{equation}
假设$a_{i j}$不变,原问题就变成了无约束最优化问题,必要条件为:
\begin{equation}
        \frac{\partial J(A, V)}{\partial v_{i}}=-\sum_{j=1}^{n} 2\left(a_{i j}\right)^{r}\left(u_{j}-v_{i}\right)=0 
\end{equation}

解之得:
\begin{equation}
    v_{i}=\frac{\sum\limits_{j=1}^{n}\left(a_{i j}\right)^{r} u_{j}}{\sum\limits_{j=1}^{n}\left(a_{i j}\right)^{r}}, \quad 1 \leqslant i \leqslant c
    \label{vij}
\end{equation}
