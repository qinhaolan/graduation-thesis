%TEX root=../main.tex
\chapter[模糊数学理论]{模糊数学理论}
\section{模糊集及其表示方法}
\subsection{模糊集的定义}
\par
在经典集合理论里面,一个集合就是某一个概念的内涵。
对于论域上的一个对象,它要么属于这个集合,要么不属于这个集合,两者只能选一个,不能两者兼之,也不能有模棱两可的情况。
而对模糊数学研究的对象来说,我们不能简单地用是或否来描述一个对象是否属于一个集合。
由此,我们把集合的特征函数的取值从$\{0,1\}$这个集合扩充到$[0,1]$这个区间上的连续取值。
越靠近1,说明该对象属于集合的程度越大,反之,越靠近0就越小。
%概括地说,模糊数学就是将现实世界的模糊概念抽象出来作为研究对象,然后用精确的数学方法探寻其中的数量规律的一门数学分支。
这样我们就把经典集合扩充到带有模糊边界的模糊集了,从而我们可以用这样的集合表示模糊概念。
\begin{definition}[模糊子集\scite{ZADEH1965fuzzy}]
    设$\mathrm{U}$为自论域,
    \[
        \mu_{\tilde{A}}: \mathrm{U} \longrightarrow[0,1]
    \]
    称$\mu$确定了$\mathrm{U}$上的一个模糊子集,记为$\tilde{A}$。
    $\mu$称为$\tilde{A}$的隶属函数,把$\mu_{\tilde{A}}(u)(u \in \mathrm{U})$的值称为$u$对于模糊子集$\tilde{A}$的隶属度。
    $\mu_{\tilde{A}}(u)$越大,代表$u$隶属于$\tilde{A}$的程度越高。
    通常,我们也把模糊子集简称为模糊集。
\end{definition}
\subsection{模糊集的表示方法}
