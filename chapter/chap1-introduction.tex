\chapter[绪论]{绪论}
%\section{绪论}
\section{研究背景}
\par
进入信息时代后,我们对信息的获取和加工能力在不断进步,我们周围的信息越来越多,五花八门各式各样的信息充斥在我们的生活中,这些信息有的是确定的,但更多的是不确定的、带有模糊性的信息.
所谓模糊性是指不确定的,介于是和不是两者之间的性质.
例如,对于优等生的判定,有的人觉得90就可以了,有的人却觉得需要达到95分以上才算优秀,所以我们很难这种非此即彼的性质去衡量一个人是不是优等生.
\par
我们所处的是一个复杂多变、时刻在运动的世界,大到星系运动,小到粒子碰撞,里面蕴含的规律都是复杂多样的.
信息本身就包含着确定性和不确定性,无所谓的好坏之分,它取决于我们如何认识信息,了解信息和使用信息.
比如,我们在评价某一个菜品时,会用“好吃”、“还行“、”难吃“来形容;描述天气时说“多云”、 “晴朗”;说一个人的衣服搭配好看等等.
这些问题很难用统一的标准去衡量,但是我们已经习惯在生活中运用模糊性所谓语言描述事物,用模糊的方法认识生活中的事物.
虽然信息带着不确定性,但是我们所处的客观世界是确定的,所以我们需要一种方法研究模糊的信息,得到清晰的结论.于是数学诞生了一个新的分支:模糊数学.
\par
1965年,L.A.Zadeh在期刊Information and Control上发表了论文《Fuzzy Sets》\cite{ZADEH1965fuzzy},标志着模糊理论的诞生.
随后,1968年,L.A.Zadeh又发表了《Probability Measures of Fuzzy Events》\cite{zadeh1968probability},进一步补充模糊理论框架.
此后模糊理论开始进入广大学者的视野,不断得到完善和改进,并在控制理论领域得到广泛应用.
\section{研究内容}
最大熵模型是一种分类学习模型,模糊熵是模糊数学里面的概念,本文在模糊理论框架下,将最大熵推广到模型信息情形,在传统的模糊C均值聚类(FCM)上进行改进,建立模糊最大熵模型并应用于实际的分类问题中,通过进一步的研究 探索模糊最大熵模型在实际问题中的应用.
\section{研究意义}
随着人工智能的大热,机器学习开始迅速应用于我们的生活中,比如商品推荐、语音识别和智能导航等.
其中,分类问题是机器学习领域的一个重要问题.
生活中许多的分类问题是模糊的,计算机无法直接处理这些模糊信息,而我们的人脑却可以很好地从这些模糊信息中得到精确的结论.
随着熵理论和模糊数学的发展,模糊数学和最大熵模型的应用范围也越来越广泛,为了处理分类问题中的不确定性,国内外的许多这方面的学者也进行了许多研究,寻找分类问题模糊性的度量方式,探寻新的实际应用.本文将模糊熵与最大熵原理结合,