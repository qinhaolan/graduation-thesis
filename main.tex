% capnocolon用于去掉图表标题中的冒号;
% titleahuart为AHU文科的章节标题格式,
% 理科请将titileahuart改为titleahuscience,
% 若要使用USTC格式请改为titlechinese;
% oneside, openany为单面选项,
% 若要双面且章节从奇数页开始,
% 请改为twoside, opentight;
\documentclass[capnocolon,titleahuscience,oneside,openany]{ahuthesis}
\usepackage{metalogo}
\usepackage{verbatim}
\usepackage{titlesec}
\usepackage{unicode-math}
%\renewcommand\thesection{\arabic {section}}
% 设置图形文件的搜索路径
\graphicspath{{chapter/}{figure/}}
%=================设置章节标题格式==================
\ctexset{
	section={
		%format用于设置章节标题全局格式,作用域为标题和编号
		%字号为小三,字体为黑体,左对齐
		%+号表示在原有格式下附加格式命令
		format+ =\heiti
	},
	subsection={
		%format用于设置章节标题全局格式,作用域为标题和编号
		%字号为小三,字体为黑体,左对齐
		%+号表示在原有格式下附加格式命令
		format+ =\heiti
	},
}

\begin{document}
%%%%%%%%%%%%%%%%%%%%%%%%%%%%%%
%% 封面部分
%%%%%%%%%%%%%%%%%%%%%%%%%%%%%%
% 封面内容
% 当中文标题过长时可以将多余的标题放在\titletail{}中
\title{模糊最大熵模型及其应用}
\entitle{Fuzzy Maximum Entropy Model and Its Application}
%全角空格可以正常输出
\author{覃浩蓝}
\enauthor{Qin Hao Lan }
\department{数学科学学院}
\No{A01714003}
\tutor{吴 涛}
\entutor{Prof. Wu Tao}
\cntime{二〇二一年 四月}
\entime{May,2021}
\major{信息与计算科学}
\enrolltime{二〇一七年 九月}
\tutordegree{教授}
\tutordepartment{数学科学学院}

% 生成USTC格式的中英文合并的扉页,AHU可以使用
%\maketitle

%生成安徽大学格式的扉页
\makeahutitle

%%%%%%%%%%%%%%%%%%%%%%%%%%%%%%
%% 前言部分
%%%%%%%%%%%%%%%%%%%%RN1%%%%%%%%%%
\frontmatter

% 题目、摘要、关键词
\begin{cnabstract}
本文先介绍了模糊数学中常用的模糊C均值(Fuzzy C-means.FCM)算法.
在模糊C均值算法中,通过与模糊数学的融合,给出了相比于K-means硬聚类更灵活的聚类结果.
在FCM算法的基础上,我们引入最大熵准则构建模糊最大熵模型,在UCI的iris数据集和seeds数据集上应用了我们的模型,并比较了它们的分类准确率.
对于非线性数据,我们的模型相对于传统的Kmeans聚类和FCM聚类有更好的表现.
  \cnkeywords{模糊集,熵,最大模糊熵,模糊聚类}
\end{cnabstract}
\renewcommand{\abstractname}{Abstract}
\begin{abstract}
  This thesis first introduces the fuzzy C-means (FCM) algorithm which is commonly used in fuzzy mathematics.
In the fuzzy C-means algorithm, by integrating with fuzzy mathematics, it gives more flexible clustering results compared with K-means hard clustering.
Based on the FCM algorithm, we introduced the maximum entropy criterion to construct the fuzzy maximum entropy model, applied our fuzzy maximum entropy model on the iris dataset and seeds dataset of UCI, and compared their classification accuracies.
For nonlinear data, our fuzzy maximum entropy model has better performance compared with traditional Kmeans clustering and FCM clustering.
  \keywords{fuzzy sets;entropy;maximum fuzzy entropy;fuzzy clustering}
\end{abstract} 
% 目录
\tableofcontents

%%%%%%%%%%%%%%%%%%%%%%%%%%%%%%
%% 正文部分
%%%%%%%%%%%%%%%%%%%%%%%%%%%%%%
\mainmatter
\chapter[绪论]{绪论}
%\section{绪论}
\section{研究背景}
\par
人类从原始社会一步步走到现在,经历了漫长的进化和发展,站到了食物链的顶端,步入了信息时代,
这一切都得益于我们对信息的获取和加工能力在不断进步。
进入信息时代后,我们周围的信息越来越多,五花八门各式各样的信息充斥在我们的生活中,这些信息有的是确定的,但更多的是不确定的、带有模糊性的信息。
所谓模糊性是指不确定的,介于是和不是两者之间的性质。
例如,对于优等生的判定,有的人觉得90就可以了,有的人却觉得需要达到95分以上才算优秀,所以我们很难这种非此即彼的性质去衡量一个人是不是优等生。
\par
我们所处的是一个复杂多变、时刻在运动的世界,大到星系运动,小到粒子碰撞,里面蕴含的规律都是复杂多样的。
信息本身就包含着确定性和不确定性,无所谓的好坏之分,它取决于我们如何认识信息,了解信息和使用信息。
比如,我们在评价某一个菜品时,会用“好吃”、“还行“、”难吃“来形容;描述天气时说“多云”、 “晴朗”;说一个人的衣服搭配好看等等。
这些问题很难用统一的标准去衡量,但是我们却可以得到清晰的结论,我们已经习惯在生活中运用模糊性所谓语言描述事物,用模糊的方法认识生活中的事物。
虽然信息带着不确定性,但是我们所处的客观世界是确定的,所以我们需要一种方法研究模糊的信息,得到清晰的结论。于是数学诞生了一个新的分支:模糊数学。
1965年,L.A.Zadeh在期刊Information and Control上发表了论文《Fuzzy Sets》,标志着模糊理论的诞生。
\section{研究内容}
最大熵模型是一种分类学习模型,模糊熵是模糊数学里面的概念,本文在模糊理论框架下,将最大熵推广到模型信息情形,在传统的模糊C均值聚类(FCM)上进行改进,建立模糊最大熵模型并应用于实际的分类问题中,通过进一步的研究 探索模糊最大熵模型在实际问题中的应用。
\section{研究意义}
随着人工智能的大热,机器学习开始迅速应用于我们的生活中,比如商品推荐、语音识别和智能导航等。
其中,分类问题是机器学习领域的一个重要问题。
生活中许多的分类问题是模糊的,计算机无法直接处理这些模糊信息,而我们的人脑却可以很好地从这些模糊信息中得到精确的结论。
随着熵理论和模糊数学的发展,模糊数学和最大熵模型的应用范围也越来越广泛,为了处理分类问题中的不确定性,国内外的许多这方面的学者也进行了许多研究,寻找分类问题模糊性的度量方式,探寻新的实际应用。本文将模糊熵与最大熵原理结合,
%TEX root=../main.tex
\chapter[模糊C-均值聚类]{模糊C-均值聚类}

\section{模糊集及其表示方法}
\par
在经典集合理论里面,一个集合就是某一个概念的内涵。
对于论域上的一个对象,它要么属于这个集合,要么不属于这个集合,两者只能选一个,不能两者兼之,也不能有模棱两可的情况。
而对模糊数学研究的对象来说,我们不能简单地用是或否来描述一个对象是否属于一个集合。
由此,我们把集合的特征函数的取值从$\{0,1\}$这个集合扩充到$[0,1]$这个区间上的连续取值。
越靠近1,说明该对象属于集合的程度越大,反之,越靠近0就越小。
%概括地说,模糊数学就是将现实世界的模糊概念抽象出来作为研究对象,然后用精确的数学方法探寻其中的数量规律的一门数学分支。
这样我们就把经典集合扩充到带有模糊边界的模糊集了,从而我们可以用这样的集合表示模糊概念。
\subsection{模糊集的定义}
\begin{definition}[模糊子集\scite{ZADEH1965fuzzy}]
    设$\mathrm{U}$为我们所研究的论域,
    \[
        \mu_{\tilde{A}}: \mathrm{U} \longrightarrow[0,1]
    \]
    称$\mu$确定了$\mathrm{U}$上的一个模糊子集,记为$\tilde{A}$。
    $\mu$称为$\tilde{A}$的隶属函数,把$\mu_{\tilde{A}}(u)(u \in \mathrm{U})$的值称为$u$对于模糊子集$\tilde{A}$的隶属度。
    $\mu_{\tilde{A}}(u)$越大,代表$u$隶属于$\tilde{A}$的程度越高。
    通常,我们也把模糊子集简称为模糊集。
\end{definition}
\subsection{模糊集的表示方法}
设有限集$\mathrm{U}=\{u_1,u_2,\dots ,u_n\}$,则有限集可以用如下几种方法表示\cite{模糊数学李安贵}。
\begin{itemize}
    \item Zadeh表示法
          \[
              \tilde{A}=\frac{\tilde{A}(u_1)}{u_1}+\frac{\tilde{A}(u_2)}{u_2}+ \dots +\frac{\tilde{A}(u_n)}{u_n}.
          \]
          虽然我们以分式和的方式表示,但是其中的$\tilde{A}(u_i)/u_i$并不表示分数,“+”也不表示和。
          $\tilde{A}(u_i)/u_i$表示的是元素$u_i$与对$\tilde{A}$的隶属度的一一对应关系;“+”表示的是$\tilde{A}$在论域$\mathrm{U}$上的整体。
    \item 序偶表示法
          \[
              \tilde{A}=\{(\tilde{A}(u_1),u_1),(\tilde{A}(u_2),u_2),\dots ,(\tilde{A}(u_n),u_n)\}.
          \]
          序偶表示法是从例举法演变而来,由元素的隶属度和对应的元素组成的有序对列出。
    \item 向量表示法
          \[
              \tilde{A}=(\tilde{A}(u_1),\tilde{A}(u_2),\dots ,\tilde{A}(u_n)).
          \]
          向量表示法是用n维数组来实现的,在论域中的元素按一定的顺序排列时,按此顺序记录元素的隶属度。
          此时也称$\tilde{A}$为模糊向量。
\end{itemize}
\section{模糊C-均值算法}
C-均值聚类是我们聚类经常用的方法之一,通过迭代计算使得目标函数达到局部最小值的时候,就是我们的最优分类。
在模糊C-均值聚类中,我们定义目标函数为:
\begin{equation}
    J(A, V)=\sum_{i=1}^{c} \sum_{j=1}^{n}\left(a_{i j}\right)^{r} d_{i j}^2
\end{equation}
$U=\{u_1,u_2,\dots,u_n\},u_j=(x_{j1},x_{j1},\dots,x_{jm})\in R^m$为给定的$n$个样本的$m$维数据集,
$A=(a_{i j})$是隶属度矩阵,$r$是模糊数,$d_{i k}=\| u_k-v_i\|$是第个$k$个样本到第$i$个聚类中心的距离。
\par
当$v_i$不变时问题等价于
\begin{equation}
    \min L(A, \lambda)=\sum_{i=1}^{c} \sum_{j=1}^{n}\left(a_{i j}\right)^{r}d_{i k}
\end{equation}
\begin{equation}
    s.t.  \sum_{i=i}^c a_{i k}=1.\  \forall k
\end{equation}
这是最优化问题,我们引入拉格朗日乘子$\lambda$,于是变为
\begin{equation}
    L(A, \lambda)=\sum_{i=1}^{c} \sum_{j=1}^{n}\left(a_{i j}\right)^{r}\left\|u_{j}-v_{i}\right\|^{2}-\sum_{j=1}^{n} \lambda_{j}\left(\sum_{i=1}^{c} a_{i j}-1\right)
    \label{Lagrange}
\end{equation}
对式\ref{Lagrange}求导,局部最小值时必要条件为
\begin{equation}
    \frac{\partial L(A, \lambda)}{\partial a_{i j}}=\left[r\left(a_{i j}\right)^{r-1}\left\|u_{j}-v_{i}\right\|^{2}-\lambda_{j}\right]=0
    \label{dLdAij}
\end{equation}
\begin{equation}
    \frac{\partial L(A, \lambda)}{\partial \lambda_{j}}=\sum_{i=1}^{c} a_{i j}-1=0
    \label{dLdLamda}
\end{equation}

由式\ref{dLdAij}可得:
\begin{equation}
    a_{i j}=\left( \frac{\lambda_j}{r \|u_{j}-v_{i}\|^2} \right)^{\frac{1}{r-1}}
    \label{daij}
\end{equation}

将式\ref{daij}带入式\ref{dLdLamda}解得:
\begin{equation}
    \left(\frac{\lambda_j}{r}\right)^{\frac{1}{r-1}}=\left[\sum\limits_{i=1}^c( \frac{1}{r \|u_{j}-v_{i}\|^2})^{\frac{1}{r-1}} \right]^{-1}
    \label{lambda}
\end{equation}
\newpage
最后将式\ref{lambda}代回式\ref{daij}得到隶属度的更新公式为:
\begin{equation}
    a_{i j}=\left[\sum\limits_{j=1}^c\left( \frac{\|u_{j}-v_{i}\|}{\|u_{j}-v_{j}\|} \right)^{\frac{2}{r-1}}\right]^{-1}
    \quad 1 \leqslant i \leqslant c,\quad 1 \leqslant j \leqslant n
    \label{aij}
\end{equation}
假设$a_{i j}$不变,原问题就变成了无约束最优化问题,必要条件为:
\begin{equation}
        \frac{\partial J(A, V)}{\partial v_{i}}=-\sum_{j=1}^{n} 2\left(a_{i j}\right)^{r}\left(u_{j}-v_{i}\right)=0 
\end{equation}

解之得:
\begin{equation}
    v_{i}=\frac{\sum\limits_{j=1}^{n}\left(a_{i j}\right)^{r} u_{j}}{\sum\limits_{j=1}^{n}\left(a_{i j}\right)^{r}}, \quad 1 \leqslant i \leqslant c
    \label{vij}
\end{equation}

\chapter[最大熵模型]{最大熵模型}
\section{模糊熵\cite{li2008entropy}}
模糊熵
\section{最大熵原理}

\section{模糊最大熵模型}
\chapter{模糊最大熵模型分类应用研究}

\section{IRIS数据集}
\par
此数据是来自UCI的鸢尾花(iris)数据集.
这是一个经典的分类数据集,由Iris Setosa(山鸢尾)、Iris Versicolour(杂色鸢尾)和Iris Virginica(维吉尼亚鸢尾)三种不同类别的鸢尾花组成,
每个样本由四个属性组成,分别是Petal.Length(花瓣长度)、Petal.Width(花瓣宽度)、Sepal.Length(花萼长度)和Sepal.Width(花萼宽度).
按两两属性绘制原始数据如下:
\begin{figure}[!ht]
    \centering
    \includegraphics[scale=0.4]{sandiantu.png}
    \caption{原始数据散点图}
    \label{散点图}
\end{figure}
\par 按4.1的算法对iris数据集进行聚类,结果如下:
\begin{table}[!ht]
    \label{聚类中心}
    \caption{聚类中心}
    \centering
    \begin{tabular}{c c c c c}
        \whline & sepal length & sepal width & petal length & petal width \\\whline
        $v_1$   & 5.0136       & 3.3903      & 1.5369       & 0.2781      \\
        $v_2$   & 6.4737       & 2.9437      & 5.1910       & 1.8012      \\
        $v_3$   & 6.0922       & 2.8186      & 4.6775       & 1.5731      \\
        \whline
    \end{tabular}
\end{table}
\begin{figure}[!ht]
    \centering
    \includegraphics[scale=0.6]{lishudu.png}
    \caption{隶属度矩阵的值}
    \label{隶属度}
\end{figure}
\newpage
与FCM算法分类结果比较
\begin{table}[!ht]
    \label{准确率比较}
    \caption{准确率比较}
    \centering
    \begin{tabular}{c | c c c c}
        \whline 算法/各类准确率 & Setosa(山鸢尾) & Versicolour(杂色鸢尾) & Virginica(维吉尼亚鸢尾) & 总样本 \\\whline
        K-means                & 100\%          & 96\%                  & 72\%                    & 89.4\%   \\
        FCM                    & 100\%          & 76\%                  & 94\%                    & 90\%   \\
        MFE                    & 100\%          & 86\%                  & 98\%                    & 94.7\% \\
        \whline
    \end{tabular}
\end{table}
从数据的比较中,可以看到,对于第一类,三个算法都做到了很好的识别,但是对于后两类非线性相关的类别,模糊最大熵模型比其他两个模型有着更好的表现.
\section{红酒数据集}


\chapter{总结与展望}
\par 
本文以模糊理论和熵理论为基础,研究了模糊熵在最大熵准则的约束下,在聚类分析中的应用,并建立了模糊最大熵模型.

在前面,我们首先介绍了模糊集和最大熵原理.独立的系统演化是一个熵增的过程,在没有外力的作用下,熵是一直增加的,即符合我们的最大熵原理.
而自然界也是一个巨大的系统,时时刻刻产生信息,有精确的,但许多都是模糊的.
进一步,为了在已有的知识下,使我们的结果更加准确,即符合自然分布,我们趋向于使得信息的熵最大,于是我们在FCM的基础上融入了最大熵模型,形成了模糊最大熵模型.
最后我们将模糊最大熵模型应用于iris数据集和seeds小麦种子的分类,取得了相对于Kmeans聚类和FCM聚类更好的聚类效果.
\par
对于下一步的探讨,我觉得是在差异因子$\beta$ 的选取上,寻找更好的算法选取差异因子$\beta$ ,而不是简简单单使用数据集的平均值代替,可能会有更好的聚类效果.
另一个思路是选取不同的模糊熵计算公式,对不同的模糊熵最大化聚类效果进行比较.
\par
现如今,适逢新一代人工智能的浪潮,各种智能算法、机器学习研究论文层出不穷,而模糊聚类在图像分割、目标识别、故障诊断等方面也有广泛的应用,
相信在不久的未来,关于模糊最大熵模型的应用与机器学习的结合会越来越多,推动模糊分类算法变得越来越好.

%%%%%%%%%%%%%%%%%%%%%%%%%%%%%%
%% 附件部分
%%%%%%%%%%%%%%%%%%%%%%%%%%%%%%
%\backmatter

% 参考文献
% 使用 BibTeX
\phantomsection
\addcontentsline{toc}{chapter}{参考文献}
\bibliography{bib/ahu}
\nocite{*} % for every item

% 致谢
\chapter{致谢}
经过几个月的努力,我终于完成了我的毕业论文,从开始到完成,每一步都是新的体验,不一样的挑战。
这个论文相对来说不够成熟,还有诸多不足之处,但是在完成它的过程中我收获良多。
我首先要感谢的是我的论文指导老师,数学科学学院的吴涛老师.吴老师在我的论文前期对我的论文研究方向给予了诸多的指导性意见,在论文撰写期间对我的论文提出了许多有益性的改建议,倾注了诸多精力。
不仅如此,吴老师还在我的生活上给予了我诸多的建议,使我受益匪浅。此外,感谢我的专业授课老师,传授我知识,感谢每一个对我的论文提供帮助的同学,有了你们才有了这篇论文的诞生。
感谢15级应用统计的胡子健学长和16级应用数学的魏兰英学姐,是他们带我走进了大学生活。
感谢我的室友,我们一起嘻嘻哈哈过了四年,建立了深厚的情谊。
感谢数院排球队的每一名队员,感谢南体排球场的球友,我在南体度过了最快乐的时光。
% 附录
\begin{appendix}
	\include{chapter/code}
\end{appendix}



\end{document}
